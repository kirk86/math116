% Figures within a column...
\makeatletter
\newenvironment{tablehere}
{\def\@captype{table}}
{}
\newenvironment{figurehere}
{\def\@captype{figure}}
{}
\makeatother

\documentclass{artikel3}
\usepackage{amssymb,amsmath,amsthm,graphicx,enumerate,textcomp,color,datetime,mathtools}
\usepackage[left=1in,top=1in,right=1in,bottom=1in]{geometry}
\usepackage{fancyhdr}
\pagestyle{fancy}
\usepackage{pgfplots}
\yyyymmdddate
\date{\formatdate{2012}{04}{13}}
\pgfplotsset{compat=1.3}
\setlength{\headheight}{15pt}
\lhead{MATH 116: Lecture Notes 6, {\thepage} of \pageref{lastpage}}
\chead{}
\rhead{\today}
\lfoot{}
\cfoot{}
\rfoot{}
\newcommand{\pr}[1]{\left(#1\right)}
\newcommand{\abs}[1]{\left|#1\right|}
\renewcommand{\vec}[1]{\underline{#1}}
\newcommand{\alt}[1]{\tilde{#1}}
\newcommand{\dotp}[2]{#1\cdot #2}
\newcommand{\norm}[1]{\left\|#1\right\|}
\newcommand{\naturals}{\mathbb{N}}
\newcommand{\integers}{\mathbb{Z}}
\newcommand{\reals}{\mathbb{R}}
\newcommand{\complex}{\mathbb{R}}
\newcommand{\rationals}{\mathbb{Q}}
\newcommand{\eqlabel}[1]{\tag{\theequation}\label{#1}\refstepcounter{equation}}
\newcommand{\mseq}[2]{#1_1,\ldots,#1_{#2}}
\newcommand{\floor}[1]{\left\lfloor #1 \right\rfloor}
\newcommand{\ceil}[1]{\left\lceil #1 \right\rceil}
\newcommand{\mvd}{\twoheadrightarrow}
\newcommand{\unit}[1]{\;\rm{#1}}
\newcommand{\del}{\partial}
\newcommand{\indentblock}[1]{\addtolength{\leftskip}{5mm}
#1
}
\newcommand{\interior}[1]{\breve{#1}}
\newtheorem{lem}{Lemma}
\newcommand{\lemref}[1]{Lemma \ref{#1}}
\newcommand{\TODO}[1]{}
\DeclareMathOperator{\diam}{diam\;}
\begin{document}
\numberwithin{equation}{enumi}
\newcommand{\eqcite}[1]{\text{[#1]}}
\newcommand{\eqnref}[1]{Stmt. \ref{#1}}
\newcommand{\expval}[1]{\left\langle #1\right\rangle}
\newcommand{\closure}[1]{\overline{#1}}
Prop. (deformation inv.).  Let $f:U\to\complex$ be a holomorphic function,
and let $\gamma_1,\gamma_2\subset U$ be two arcs.  Then \[
	\int_{\gamma_1}f(z)dz=\int_{\gamma_2}f(z)dz
\] if one of the following holds:
\begin{enumerate}
	\item
		Both $\gamma_1$ and $\gamma_2$ are closed, and one is a (continuous) deformation
		of the other.
	\item
		Both $\gamma_1$ and $\gamma_2$ are arcs from $\alpha$ to $\beta$, and one
		is a deformation of the other (fixing the endpoints).
\end{enumerate}

Remark.  When talking about $U$, $\gamma_1$ deformed to
$\gamma_2$, etc., always consider a deformation contained entirely in $U$.  For
instance, one cannot deform across a hole in the domain $U$.

Proof.  First, prove claim 1; 2 is very similar.

Key technique: Consider curve $\gamma\subset U$, and a slight
perturbation $\gamma'\subset U$ near some point $a$.  Then, we can define $\gamma'=\gamma+\Gamma$
where $\Gamma$ is a small loop containing $\gamma'$.  Since $\Gamma$ is small,
consider first the case where $\Gamma$ lies entirely in some disc $D\subset U$.
Then (because $f$ is holomorphic) $f$ has an antiderivative $F$ on $D$,
and $\int_\Gamma f(z)dz=F(\Gamma(b))-F(\Gamma(a))=0$, because
$\Gamma$ is closed.

Accumulating such perturbations over all the points of the curve, any continuous
deformation can be achieved.

Proof.  We call $\gamma_1,\gamma_2$ (both closed) sufficiently close if there are closed
arcs $\Gamma_1,\ldots,\Gamma_N$ such that
\begin{enumerate}
	\item
		$\gamma_1-\gamma_2=\Gamma_1+\cdots+\Gamma_N$,
	\item
		each $\Gamma_i\subset D(a_i,r_i)\subset U$ for some $a_i\in U$, $r_i>0$.
\end{enumerate}

Over the discs $D(a_i,r_i)$ we find holomorphic antiderivatives
$F_i:D(a_i,r_i)\to\complex$ such that $F_i'=f\mid_{D(a_i,r_i)}$.
Then \[
	\int_{\Gamma_i} f(z)dz=\int_{\Gamma_i}F_i'(z)dz=0
\] (i.e. int. of derivative of holomorphic function over closed arc is zero).

This means that \[
	\int_{\gamma_1}f(z)dz = \int_{\gamma_2}f(z)dz+\sum_{i=1}^{n}\int_{\Gamma_i}f(z)dz
		= \int_{\gamma_2}f(z)dz.
\]

In general, suppose $\gamma_1,\gamma_2$ are closed and $\gamma_2$ is a deformation
of $\gamma_1$ within $U$.  Then we can find closed arcs $\alt{\gamma}_1,\alt{\gamma}_2,\ldots,\alt{\gamma}_n$
such that $\gamma_1=\alt{\gamma}_1$, $\gamma_2=\alt{\gamma}_n$, and $\alt{\gamma}_{k+1}$
is suff. close to $\alt{\gamma}_k$ for $k=1,2,\ldots,n-1$ (i.e. we can break up the continuous deformation
into finitely many sufficiently close perturbations).  Note that this would immediately
imply the desired equality (progressing through each of the sufficiently close perturbations,
i.e. \[
	\int_{\gamma_1=\alt{\gamma}_1}f = \int_{\alt{\gamma}_2}f=\cdots =\int_{\alt{\gamma}_n=\gamma_2}f.
\]

Note: consider two curves $\gamma_1,\gamma_2:[0,1]\to U$ where $\gamma_1(0)=\gamma_1(1)$,
$\gamma_2(0)=\gamma_2(1)$.  Then $\gamma_1$ deforms to $\gamma_2$ iff there is some continuous
function $\Gamma:[0,1]^2\to U$ such that $\gamma_1(t)=\Gamma(0,t)$, $\gamma_2(t)=\Gamma(1,t)$,
and $\Gamma(s,0)=\Gamma(s,1)$ for any $s$ (i.e. for every $s$, $\Gamma(s,\cdot)$ is a closed curve).

Bonus: Note then that the image of $\Gamma$ is another compact set, so there is some $\epsilon>0$ such
that for any $\Gamma(s,t)$ in the image of $\Gamma$, an $\epsilon$-neighborhood of $\Gamma(s,t)$
lies entirely in $U$.  Then, note that $\Gamma$ is continuous on a compact set, and is thus uniformly
continuous, so there is some $\delta>0$ such that if $\abs{s'-s}<\delta$ and $\abs{t'-t}<\delta$,
then $\abs{Gamma(s',t')-\Gamma(s,t)}<\epsilon$.  Take $\frac{1}{n}<\delta$, and $\Gamma(0,\cdot),\Gamma(1/n,\cdot),\ldots,\Gamma(n/n,\cdot)$
will be the desired family of sufficiently close curves.  To extend to the case of non-closed curves
with endpoints $\alpha,\beta$, fix $\Gamma(s,0)=\alpha$ and $\Gamma(s,1)=\beta$ instead of
$\Gamma(s,0)=\Gamma(s,1)$.

Theorem (Cauchy integral formula).  Let $f:U\to\complex$ be holomorphic, and let $D(a,r)$
be a disc such that its closure $\overline{D(a,r)}\subset U$.  Then for any $z_0\in D(a,r)$, \[
	f(z_0)=\frac{1}{2\pi i}\oint_{\abs{z-a}=r}\frac{f(z)}{z-z_0}dz.
\]

Proof: We want to evaluate the integral \[
	\oint_{\abs{z-a}=r}\frac{f(z)}{z-z_0}dz.
\] We look at $F(z)=\frac{f(z)}{z-z_0}$, where $F$ is
defined on $U\setminus\{z_0\}$ which is holomorphic over its domain.
Note that the integral of $F$ over a circle is invariant up to deformation
in $U\setminus\{z_0\}$; in particular, we can deform the circle into an infinitesimal
loop about $z_0$.  Then, note that for any $\epsilon>0$, because $f$ is continuous,
there is some $\delta>0$ such that $\abs{f(z)-f(z_0)}<\epsilon$ when $\abs{z-z_0}\leq\delta$.
Then \[
	\oint_{\abs{z-a}=r}\frac{f(z)}{z-z_0}dz
	= \oint_{\abs{z-z_0}=\delta}\frac{f(z)}{z-z_0}dz.
\] Parametrizing the curve by $z=z_0+\delta e^{i\theta}$ for $\theta\in [0,2\pi]$, \[
	\oint_{\abs{z-z_0}=\delta}\frac{f(z)}{z-z_0}dz
	= \int_{0}^{2\pi}\frac{f(z_0+\delta e^{i\theta})}{\delta e^{i\theta}}\delta ie^{i\theta}d\theta
	= i\int_{0}^{2\pi}f(z_0+\delta e^{i\theta})d\theta.
\] Then, note that \[
	\abs{i\int_{0}^{2\pi}f(z_0+\delta e^{i\theta})d\theta-2\pi if(z_0)}
	= \abs{\int_{0}^{2\pi}(f(z_0+\delta e^{i\theta})-f(z_0))d\theta}
	\leq \int_{0}^{2\pi}\abs{f(z_0+\delta e^{i\theta})-f(z_0)}d\theta
	\leq \int_{0}^{2\pi}\epsilon d\theta=2\pi\epsilon,
\] so the integral approaches $2\pi if(z)$ as $\delta\to 0$.  Thus, \[
	2\pi if(z)=\lim_{\delta\to 0}\oint_{\abs{z-a}=r}\frac{f(z)}{z-z_0}dz
	= \oint_{\abs{z-a}=r}\frac{f(z)}{z-z_0}dz,
\] since the integral is independent of $\delta$, and thus \[
	f(z)=\frac{1}{2\pi i}\oint_{\abs{z-a}=r}\frac{f(z)}{z-z_0}dz.
\]

Application.  Evaluate: \[
	\oint_{\abs{z}=2}\frac{z^2+3z}{(z-3)(z+1)}dz.
\] Note that $\frac{z^2+3z}{(z-3)(z+1)}$ is holomorphic
on $\complex\setminus\{-1,3\}$.  Thus, over the disc $\abs{z}\leq 2$,
the function is holmorphic except at $z=-1$, so we can deform the loop to \[
	\oint_{\abs{z}=2}\frac{z^2+3z}{(z-3)(z+1)}dz
	= \oint_{\abs{z+1}=1}\frac{z^2+3z}{(z-3)(z+1)}dz
	= \oint_{\abs{z-(-1)}=1}\frac{(z-3)^{-1}(z^2+3z)}{(z-(-1))}dz.
\] Using the integral formula, \[
	\oint_{\abs{z-(-1)}=1}\frac{(z-3)^{-1}(z^2+3z)}{(z-(-1))}dz
	= 2\pi i (-1-3)^{-1}((-1)^2+3(-1))
	= 2\pi i\frac{1-3}{-4} = \pi i.
\] 

Now, consider \[
	\oint_{\abs{z}=2}\frac{z^2+3z}{(z-1)(z+1)}dz.
\] This can be done similarly, except by deforming the large circle into
two smaller circles surrounding the two discontinuities $z=1$ and $z=-1$.

In general, \[
	f(z)= \frac{1}{2\pi i}\oint_{\abs{w-a}=r}\frac{f(w)}{w-z}dw.
\] Next time, will show that \[
	f'(z)= \frac{1}{2\pi i}\oint_{\abs{w-a}=r}\frac{f(w)}{(w-z)^2}dw,
\] and \[
	f^{(n)}(z)= \frac{n!}{2\pi i}\oint_{\abs{w-a}=r}\frac{f(w)}{(w-z)^{n+1}}dw
\]
\label{lastpage}
\end{document}
