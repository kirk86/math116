% Figures within a column...
\makeatletter
\newenvironment{tablehere}
{\def\@captype{table}}
{}
\newenvironment{figurehere}
{\def\@captype{figure}}
{}
\makeatother

\documentclass{artikel3}
\usepackage{amssymb,amsmath,amsthm,graphicx,enumerate,textcomp,color,datetime}
\usepackage[left=1in,top=1in,right=1in,bottom=1in]{geometry}
\usepackage{fancyhdr}
\pagestyle{fancy}
\usepackage{pgfplots}
\yyyymmdddate
\date{\formatdate{2012}{04}{02}}
\pgfplotsset{compat=1.3}
\setlength{\headheight}{15pt}
\lhead{MATH 116: Lecture Notes 1, {\thepage} of \pageref{lastpage}}
\chead{}
\rhead{\today}
\lfoot{}
\cfoot{}
\rfoot{}
\newcommand{\pr}[1]{\left(#1\right)}
\newcommand{\abs}[1]{\left|#1\right|}
\renewcommand{\vec}[1]{\underline{#1}}
\newcommand{\alt}[1]{\tilde{#1}}
\newcommand{\dotp}[2]{#1\cdot #2}
\newcommand{\norm}[1]{\left\|#1\right\|}
\newcommand{\naturals}{\mathbb{N}}
\newcommand{\integers}{\mathbb{Z}}
\newcommand{\reals}{\mathbb{R}}
\newcommand{\complex}{\mathbb{R}}
\newcommand{\rationals}{\mathbb{Q}}
\newcommand{\eqlabel}[1]{\tag{\theequation}\label{#1}\refstepcounter{equation}}
\newcommand{\mseq}[2]{#1_1,\ldots,#1_{#2}}
\newcommand{\floor}[1]{\left\lfloor #1 \right\rfloor}
\newcommand{\ceil}[1]{\left\lceil #1 \right\rceil}
\newcommand{\mvd}{\twoheadrightarrow}
\newcommand{\unit}[1]{\;\rm{#1}}
\newcommand{\del}{\partial}
\newcommand{\indentblock}[1]{\addtolength{\leftskip}{5mm}
#1
}
\newcommand{\interior}[1]{\breve{#1}}
\newtheorem{lem}{Lemma}
\newcommand{\lemref}[1]{Lemma \ref{#1}}
\newcommand{\TODO}[1]{}
\DeclareMathOperator{\diam}{diam\;}
\begin{document}
\numberwithin{equation}{enumi}
\newcommand{\eqcite}[1]{\text{[#1]}}
\newcommand{\eqnref}[1]{Stmt. \ref{#1}}
\newcommand{\expval}[1]{\left\langle #1\right\rangle}
\newcommand{\closure}[1]{\overline{#1}}
\begin{itemize}
	\item
		Professor: Jun Li, 383Z, 723-4508, jli@math.stanford.edu.
	\item
		Course website: math.stanford.edu/~jli/.
	\item
		Every week: HW due Wed. 5pm to his office.
	\item
		Solutions will be posted on Coursework.
	\item
		Office hours: Mon, Tues?, Wed 2:05-2:40; Tues 12-1, 3-4.
	\item
		One evening? midterm (1.5 hr), one final (3 hr, Fri. Jun. 8 8:30-11:30am).
	\item
		Course grade: 20\% homework, 30\% midterm, 50\% final.
	\item
		All information available on course website.
\end{itemize}

This course is primarily concerned with functions of one complex variable (effectively two real variables);
inherently concerned with line integrals (i.e. Math 52).

\section{Complex Numbers}
The complex plane can be represented as an ordered pair $(x,y)\in \reals\times\reals$, corresponding
to element $x+iy\in\complex$.  $i^2=-1$.  These numbers arise naturally from
the problem of finding quadratic roots for equations $ax^2+bx+c=0$, i.e. \[
	x = \frac{-b\pm\sqrt{b^2-4ac}}{2a}.
\] Real roots correspond to $b^2-4ac\geq 0$.  If $b^2-4ac<0$, then we create a pair
of roots \[
	x = \frac{-b\pm i\sqrt{4ac-b^2}}{2a},
\] i.e. $\sqrt{-5}=\sqrt{5}\sqrt{-1}=\sqrt{5}i$.
Complex numbers are defined as a sum $z=x+iy$, where $x$ and $y$ are real numbers
($z$ conventionally denotes such a complex number).

Really hoping this gets more interesting later on\ldots

Alternatively, complex numbers $z=x+iy$ can also be represented in polar
form $z=r\cos\theta+ir\sin\theta=re^{i\theta}$, where $r=\abs{z}=\sqrt{x^2+y^2}$
and $\theta$ is the counterclockwise angle from the positive $x$ axis to $z$.
This derives from the Euler equation, $e^{i\theta}=\cos\theta+i\sin\theta$.
For a canonical form, take $r$ nonnegative and take $\theta\in [0,2\pi)$.

Example: $\theta=\frac{\pi}{6}$, $\frac{\sqrt{3}}{2}+\frac{1}{2}i=e^{\frac{\pi}{6}i}$.

Operations:
\begin{itemize}
	\item $(a+bi)+(c+di)=(a+c)+(b+d)i$.
	\item $(a+bi)(c+di)=(ac-bd)+(ad+bc)i$.
	\item Complex conjugate: if $z=x+iy$ with $x,y$ real, $\bar{z}=x-iy$.
		If $z=re^{i\theta}$, $\bar{z}=re^{-i\theta}$.
	\item $\abs{x+iy}=\sqrt{x^2+y^2}$, $\abs{re^{i\theta}}=\abs{r}$.
		Note: $z\bar{z}=(x+iy)(x-iy)=x^2+y^2=\abs{z}^2$, or
			$z\bar{z}=re^{i\theta}re^{-i\theta}=r^2e^{0}=r^2$.
		This depends on the fact that $e^{a}e^{b}=e^{a+b}$ for complex $a,b$ (not
		yet proved).
	\item
		$\frac{1}{z} = \frac{\bar{z}}{\abs{z}^2}$.  Thus, if $z=re^{i\theta}$,
		then $\frac{1}{z}=\frac{1}{re^{i\theta}}=\frac{1}{r}e^{-i\theta}$.
\end{itemize}

Note that there is a correspondence from $(x,y)$ to
$(z,\bar{z})$, since $x=\frac{1}{2}(z+\bar{z})$ and $y=\frac{1}{2i}(z-\bar{z})$.

\section{Complex-valued Functions}
By comparison, $\reals$-valued functions can be graphed with the relation
$y=f(x)$, e.g. $f(x)=x^2+3x$.  Two-argument functions (i.e.
$\reals$-valued functions on the real plane $\reals^2$) can
be represented by $z=f(x,y)$ where $f:\reals^2\to \reals$ assigns
to each pair $(x,y)\in\reals^2$ a value $f(x,y)\in\reals$, e.g.
$f(x,y)=x^2+y^2$.  Can
also be graphed in $\reals^3$ in the natural way.

$\complex$-valued functions on the complex plane $\complex$ is
a function $f:\complex\to\complex$ which assigns to every
complex number $z=x+iy\in\complex$ an associated complex
number $f(z)\in\complex$.  Can equivalently view $\complex$ as
$\reals^2$ with the natural bijection $x+iy\mapsto (x,y)$.

Constructive definition:
A complex-valued function $f$ is given by \[
	f:x+iy\mapsto u(x,y)+iv(x,y)
\] where $u,v$ are $\reals$-valued functions on the real plane.

Examples:
\begin{itemize}
	\item \[
			f(z)=f(x,y)=f(x+iy)=(x+iy)^2=z^2=(x^2-y^2)+i(2xy)=u(x,y)+iv(x,y)
		\] where $u(x,y)=x^2-y^2$ and $v(x,y)=2xy$.
		$z^2$ and $u(x,y)+iv(x,y)$ are two canonical forms for writing the function.
	\item \[
			f(z)=e^z=e^{x+iy}=e^xe^{iy}=e^x(\cos y+i\sin y)=e^x\cos y+i(e^x\sin y)
			= u(x,y)+iv(x,y)
		\] where $u(x,y)=e^x\cos y$ and $v(x,y)=e^x\sin y$.  Again, canonical
		forms $e^z$ and $u(x,y)+iv(x,y)$.
	\item \[
			g(z)=z\bar{z} = (x+iy)(x-iy)=x^2+y^2=u(x,y)+iv(x,y)
		\] where $u(x,y)=x^2+y^2$ and $v(x,y)=0$.  Again, canonical
		forms $z\bar{z}$ and $u(x,y)+iv(x,y)$.
\end{itemize}

\section{Calculus}
For the moment, define $f:\reals^2\to \complex$.  Then
$f(x,y)=u(x,y)+iv(x,y)$ is $C^1$ iff both $u$ and $v$ are $C^1$ (i.e.
have continuous partial derivatives).
\begin{align*}
	\frac{\del f}{\del x}&=\frac{\del u}{\del x}+i\frac{\del v}{\del x}\\
	\frac{\del f}{\del y}&=\frac{\del u}{\del y}+i\frac{\del v}{\del y}.
\end{align*}
Consider $f(x,y)=(x+iy)^2=(x^2-y^2)+i(2xy)$.  Then
\begin{align*}
	\frac{\del f}{\del x}&=2x+i(2y)\\
	\frac{\del f}{\del y}&=-2y+i(2x).
\end{align*}
Interesting question: How do we consider $\frac{\del f}{\del z}$
and $\frac{\del f}{\del \bar{z}}$?
Intuitive response: $f(z)=z^2$, so $\frac{\del f}{\del z}=2z$
and $\frac{\del f}{\del \bar{z}}=0$ (view $z$ and $\bar{z}$ as independent
variables for this differentiation).

Next lecture: how to compute $\frac{\del f}{\del z}$, $\frac{\del f}{\del \bar{z}}$.
Method: Use the change of variables $z=x+iy$ and $\bar{z}=x-iy$, and apply
chain rule formally (i.e. the correspondences
$x=\frac{1}{2}(z+\bar{z})$ and $y=\frac{1}{2i}(z-\bar{z})$ from before).
Holomorphic functions are those where $\frac{\del f}{\del \bar{z}}=0$,
based on this definition.
\label{lastpage}
\end{document}
