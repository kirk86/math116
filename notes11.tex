% Figures within a column...
\makeatletter
\newenvironment{tablehere}
{\def\@captype{table}}
{}
\newenvironment{figurehere}
{\def\@captype{figure}}
{}
\makeatother

\documentclass{artikel3}
\usepackage{amssymb,amsmath,amsthm,graphicx,enumerate,textcomp,color,datetime,mathtools}
\usepackage[left=1in,top=1in,right=1in,bottom=1in]{geometry}
\usepackage{fancyhdr}
\pagestyle{fancy}
\usepackage{pgfplots}
\yyyymmdddate
\date{\formatdate{2012}{04}{25}}
\pgfplotsset{compat=1.3}
\setlength{\headheight}{15pt}
\lhead{MATH 116: Lecture Notes 11, {\thepage} of \pageref{lastpage}}
\chead{}
\rhead{\today}
\lfoot{}
\cfoot{}
\rfoot{}
\newcommand{\pr}[1]{\left(#1\right)}
\newcommand{\abs}[1]{\left|#1\right|}
\renewcommand{\vec}[1]{\underline{#1}}
\newcommand{\alt}[1]{\tilde{#1}}
\newcommand{\dotp}[2]{#1\cdot #2}
\newcommand{\norm}[1]{\left\|#1\right\|}
\newcommand{\naturals}{\mathbb{N}}
\newcommand{\integers}{\mathbb{Z}}
\newcommand{\reals}{\mathbb{R}}
\newcommand{\complex}{\mathbb{R}}
\newcommand{\rationals}{\mathbb{Q}}
\newcommand{\eqlabel}[1]{\tag{\theequation}\label{#1}\refstepcounter{equation}}
\newcommand{\mseq}[2]{#1_1,\ldots,#1_{#2}}
\newcommand{\floor}[1]{\left\lfloor #1 \right\rfloor}
\newcommand{\ceil}[1]{\left\lceil #1 \right\rceil}
\newcommand{\mvd}{\twoheadrightarrow}
\newcommand{\unit}[1]{\;\rm{#1}}
\newcommand{\del}{\partial}
\newcommand{\indentblock}[1]{\addtolength{\leftskip}{5mm}
#1
}
\newcommand{\interior}[1]{\breve{#1}}
\newtheorem{lem}{Lemma}
\newtheorem{thm}{Theorem}
\newtheorem{cor}{Corollary}
\newtheorem{prop}{Proposition}
\newcommand{\lemref}[1]{Lemma \ref{#1}}
\newcommand{\TODO}[1]{}
\DeclareMathOperator{\diam}{diam\;}
\numberwithin{equation}{enumi}
\newcommand{\eqcite}[1]{\text{[#1]}}
\newcommand{\eqnref}[1]{Stmt. \ref{#1}}
\newcommand{\expval}[1]{\left\langle #1\right\rangle}
\newcommand{\closure}[1]{\overline{#1}}
\begin{document}
Proposed midterm time: 6th week Friday, or 7th week Monday,
12:30-2pm.

Review, Cauchy integral formula:
\begin{thm}
	If $f:D(a,r+\epsilon)\to \complex$ is holomorphic (for $\epsilon>0$),
	then for any $z\in D(a,r)$, \[
		f(z)=\frac{1}{2\pi i}\oint_{\abs{w-a}=r}\frac{f(w)}{w-z}dw.
	\] Warning: this is NOT true if $f$ is not holomorphic.
\end{thm}

\begin{thm}
	Riemann extension theorem:
	Given $f:D(a,r+\epsilon)^*\to\complex$
	holomorphic, $\lim_{z\to a}f(z)$ exists,
	then $f$ extends to a holomorphic function of $D(a,r+\epsilon)$.
\end{thm}
\begin{proof}
	Approach: define \[
		\alt{f}(z)=\frac{1}{2\pi i}\oint_{\abs{w-a}=r}\frac{f(w)}{w-z}dw.
	\] Since $f(w)$ is $C^0$ on $\abs{w-a}=r$, we already know
	$\alt{f}$ is holomorphic on $D(a,r)$.  We need only show
	that $\alt{f}=f$ on $D(a,r)^*$ (recall that $D(a,r)^*=D(a,r)\setminus\{a\}$
	is the punctured disk).
	
	Pick $z\in D(a,r)^*$.  By invariance of integration, we can
	shrink the loop $\abs{w-a}=r$ to two loops, one $\abs{w-a}=\delta$ for very
	small $\delta$, and a second loop $\abs{w-z}=\delta$ (note
	that the function $\frac{f(w)}{w-z}$ as a function of $w$
	is holomorphic when $w\neq z,a$, i.e. over $D(a,r+\epsilon)\setminus\{a,z\}$).
	
	Thus, \[
		\frac{1}{2\pi i}\oint_{\abs{w-z}=\delta}\frac{f(w)}{w-z}dw
		= \frac{1}{2\pi i}\oint_{\abs{w-a}=r}\frac{f(w)}{w-z}dw
		- \frac{1}{2\pi i}\oint_{\abs{w-a}=\delta}\frac{f(w)}{w-z}dw,
	\] so \[
		f(z)=\alt{f}(z)-\frac{1}{2\pi i}\oint_{\abs{w-a}=\delta}\frac{f(w)}{w-z}dw.
	\] By taking $\delta\to 0$, the second integral goes to zero (because $\abs{f(z)}$
	is bounded in some neighborhood of $z=a$).  Thus $f(z)=\alt{f}(z)$.
\end{proof}

Future approach: though the Cauchy integral formula fails when there are
singularities in the domain, we can instead integrate over the loop and subtract
loop integrals around the singularities (i.e. add clockwise loop integrals
around the singularities) to retrieve the value of $f(z)$.

\section{Isolated singularities}
Definition.  We say $a$ is an isolated singularity of $f$ if
there is an $r>0$ such that $f$ is holomorphic on $D(a,r)^*$.

Cases:
\begin{enumerate}
	\item
		$\lim_{z\to a}f(z)=\alpha\in\complex$ exists.
	\item
		$\lim_{z\to a}\abs{f(z)}=\infty$ (written as
		$\lim_{z\to a}f(z)=\infty$, i.e. unbounded).
	\item
		$\lim_{z\to a}f(z)$ does not exist at all (even
		in $\complex\cup\{\infty\}$).
\end{enumerate}

Examples:
\begin{enumerate}
	\item
		\[
			\frac{\sin z}{z}=\frac{1}{z}\pr{z-\frac{z^3}{6}+\cdots}=1-\frac{z^2}{6}+\cdots
		\] is holomorphic near zero, but by Riemann extension theorem this function
		is holomorphic everywhere.
	\item
		\[
			\frac{\sin z}{z^2}=\frac{1}{z}-\frac{z}{6}+\cdots
		\] is holomorphic except at $z=0$, where it becomes unbounded (in all directions).
	\item
		For the function $e^{1/z}$, the limit as $z\to 0$ does not exist.
\end{enumerate}

\begin{prop}
	Suppose $f$ has an isolated singularity at $z=a$, and $\lim_{z\to a}f(z)=\infty$.
	Then \[
		f(z)=\frac{g(z)}{(z-a)^k}
	\] for some positive integer $k$, where $g(z)$ holomorphic near $z=a$,
	and $g(a)\neq 0$.
\end{prop}
\begin{proof}
	One possible strategy: look at $(z-a)^kf(z)$; but no bound on which $k$ we need.
	Other strategy: look at $\frac{1}{f(z)}=\frac{(z-a)^k}{g(z)}$.  Note
	that since $\lim_{z\to a}f(z)=\infty$, for sufficiently small neighborhood
	$D(a,\epsilon)^*$ will have no zeros of $f$.  Then $h(z)=\frac{1}{f(z)}$
	is holomorphic on $D(a,\epsilon)^*$.  Apply the Riemann extension theorem
	to $h$, since $h(z)\to 0$ as $z\to a$.  Then $h(z)=(z-a)^k\alt{h}(z)$
	where $\alt{h}$ holomorphic, $\alt{h}(a)\neq 0$, $k\geq 1$
	(since $h(a)=0$).  Then set $g(z)=\frac{1}{\alt{h}(z)}$ to complete
	the proof.
\end{proof}

Definition.  We call case 1 (i.e. where finite limit exists) a removable
singularity.  We call case 2 (where the limit is infinity) a pole of
order $k$ (of $f$).  In particular, $k$ is the positive integer
such that $(z-a)^kf(z)$ has a nonzero removable singularity at $a$.
We call case 3 an essential singularity (note: these are the singularities
where the function becomes truly wild).

\begin{thm}
	Casorati-Weierstrass: Suppose $z=a$ is an essential singularity
	of $f$.  Then for any (sufficiently small) $\epsilon>0$, $f(D(a,\epsilon)^*)$
	dense in $\complex$.
\end{thm}
Remark: $e^{1/z}$ has an essential singularity at $z=0$, so for any $\alpha\in \complex\cup\{\infty\}$
and any $\epsilon>0$,
there is a sequence $p_n\in D(0,\epsilon)$ such that $e^{1/p_n}\to \alpha$.

Second remark: there is a proof that the image
of such a punctured disk covers the entire complex
plane except possibly two points, but not three (not covered in class).

\begin{proof}
	Suppose $\alpha\notin \overline{f(D(a,\epsilon)^*)}$.  Then
	there is some $\delta>0$ such that $D(\alpha,\delta)\cap f(D(a,\epsilon)^*)=\emptyset$.
	
	We introduce \[
		g(z)=\frac{z-a}{f(z)-\alpha}.
	\] $f$ has an isolated singularity at $a$, so $g$ is holomorphic
	over $D(a,\epsilon)^*$ except where $f(z)=\alpha$, but by assumption
	this does not occur (see above).  Thus $g$ has an isolated
	singularity at $z=a$, but \[
		\lim_{z\to a}g(z)=0,
	\] because \[
		\abs{\frac{1}{f(z)-\alpha}}\leq\frac{1}{\delta}
	\] for $z\in D(a,\epsilon)^*$.  Thus $g$ is holomorphic near $a$, and \[
		f(z)=\frac{z-a}{g(z)}+\alpha.
	\] This function has at most a pole at $z=a$, since $g$ is holomorphic
	near $a$ (i.e. use Riemann extension theorem and factoring zeros).
\end{proof}

Remark: this implies that if a holomorphic function with an isolated
singularity is bounded near the point, it must have a limit
near the point (i.e. it is neither a pole nor an essential singularity).

Examples:
\begin{enumerate}
	\item
		$\frac{1}{z(1-z)}$ has two poles at $0,1$ of order $1$.
	\item
		$\frac{1}{z}\sin z$ has a removable singularity.
	\item
		$\frac{1}{z}\cos z$ has a pole of order $1$.
	\item
		$\pr{z+\frac{1}{z}}^2$ has a pole of order $2$.
	\item
		$e^{1/z^2}$ has an essential singularity.  To prove,
		take $\frac{1}{z_n^2}=2n\pi i$, then $e^{1/z_n^2}\to 1$
		but $\frac{1}{w_n^2}=2n\pi i+1$ implies $e^{1/w_n^2}\to e$,
		and $\lim_{z\to 0}e^{1/z^2}$ does not exist (in $\complex\cup\{\infty\}$).
\end{enumerate}

To prove $e^z+e^{1/z}$ is holomorphic on $\complex^*=\complex\setminus\{0\}$, then \[
	e^z+e^{1/z}=\cdots+\frac{1}{n!}z^{-n}+\cdots+z^{-1}+2+z+\cdots+\frac{1}{n!}z^n+\cdots.
\] This is called a Laurent series (which converges on an annulus; in this case,
on the annulus $0<\abs{z}<\infty$), and next time: will prove that if there are an infinite
number of negative powers in this series, then the singularity is essential; if there
are only finitely many, the singularity is a pole.
\label{lastpage}
\end{document}
