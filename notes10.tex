% Figures within a column...
\makeatletter
\newenvironment{tablehere}
{\def\@captype{table}}
{}
\newenvironment{figurehere}
{\def\@captype{figure}}
{}
\makeatother

\documentclass{artikel3}
\usepackage{amssymb,amsmath,amsthm,graphicx,enumerate,textcomp,color,datetime,mathtools}
\usepackage[left=1in,top=1in,right=1in,bottom=1in]{geometry}
\usepackage{fancyhdr}
\pagestyle{fancy}
\usepackage{pgfplots}
\yyyymmdddate
\date{\formatdate{2012}{04}{23}}
\pgfplotsset{compat=1.3}
\setlength{\headheight}{15pt}
\lhead{MATH 116: Lecture Notes 10, {\thepage} of \pageref{lastpage}}
\chead{}
\rhead{\today}
\lfoot{}
\cfoot{}
\rfoot{}
\newcommand{\pr}[1]{\left(#1\right)}
\newcommand{\abs}[1]{\left|#1\right|}
\renewcommand{\vec}[1]{\underline{#1}}
\newcommand{\alt}[1]{\tilde{#1}}
\newcommand{\dotp}[2]{#1\cdot #2}
\newcommand{\norm}[1]{\left\|#1\right\|}
\newcommand{\naturals}{\mathbb{N}}
\newcommand{\integers}{\mathbb{Z}}
\newcommand{\reals}{\mathbb{R}}
\newcommand{\complex}{\mathbb{R}}
\newcommand{\rationals}{\mathbb{Q}}
\newcommand{\eqlabel}[1]{\tag{\theequation}\label{#1}\refstepcounter{equation}}
\newcommand{\mseq}[2]{#1_1,\ldots,#1_{#2}}
\newcommand{\floor}[1]{\left\lfloor #1 \right\rfloor}
\newcommand{\ceil}[1]{\left\lceil #1 \right\rceil}
\newcommand{\mvd}{\twoheadrightarrow}
\newcommand{\unit}[1]{\;\rm{#1}}
\newcommand{\del}{\partial}
\newcommand{\indentblock}[1]{\addtolength{\leftskip}{5mm}
#1
}
\newcommand{\interior}[1]{\breve{#1}}
\newtheorem{lem}{Lemma}
\newtheorem{thm}{Theorem}
\newtheorem{cor}{Corollary}
\newcommand{\lemref}[1]{Lemma \ref{#1}}
\newcommand{\TODO}[1]{}
\DeclareMathOperator{\diam}{diam\;}
\numberwithin{equation}{enumi}
\newcommand{\eqcite}[1]{\text{[#1]}}
\newcommand{\eqnref}[1]{Stmt. \ref{#1}}
\newcommand{\expval}[1]{\left\langle #1\right\rangle}
\newcommand{\closure}[1]{\overline{#1}}
\begin{document}
11f. Consider \[
	f(z)=\sum_{k=0}^{\infty} \frac{k}{k^2+4}z^k.
\] For $z=1$, this sum diverges; for $z=-1$, this sum converges.
For $z=e^{i\theta}$, $\theta\neq 0,\pi$, do as much as you can.

14. Circle?

54. Very involved.

Continued from previous lecture?:

Consider $f:U\to\complex$ with $U$ connected, $a\in U$.
\begin{lem}
	Suppose $f(a)=0$ and $f^{(n)}(a)\neq 0$ for some $n$.  Then there
	is some $r>0$ such that $D(a,r)\subset U$ and $f(z)$ has no
	zeros in $D(a,r)\setminus \{a\}$.
\end{lem}

Define $D(a,r)^*=D(a,r)\setminus\{a\}$.

\begin{proof}
	Take $f$ holomorphic over $\overline{D(a,R)}\subset U$.
	Expanding at $a$, \[
		f(z)=f(a)+\cdots+\frac{f^{(k)}(a)}{k!}(z-a)^k+\cdots
	\] where $f^{(k)}(a)$ is the first nonzero derivative of $f$
	at $a$; this exists because there is at least one nonzero
	derivative, as assumed.
	
	Then \[
		g(z)=\sum_{\ell=k}^{\infty}\frac{f^{(\ell)}(a)}{\ell!}(z-a)^{\ell-k},
	\] and \[
		f(z)=(z-a)^kg(z).
	\] Note that $g(a)=\frac{f^{(k)}(a)}{k!}\neq 0$.
	
	Since $g(z)$ converges in a small disk $D(a,r)$ and is holomorphic,
	and since $g(a)\neq 0$, then there exists some small $r\leq R$
	such that $g(z)$ has no zeros in $D(a,r)^*$.
\end{proof}

\begin{cor}
	If $f^{(n)}(a)=0$ for all $n\geq 0$, then $f(z)=0$ in some small
	disk $D(a,r)$.
\end{cor}

Now, this raises the question: can a holomorphic
function $f$ be zero over some small disk, but nevertheless
be nonzero somewhere else over the same connected
region $U$? Answer: No!

Take \[
	\Sigma_n=\{p\in U:f^{(n)}(p)=0\}.
\] Note that this is closed, because each $f^{(n)}$ is
continuous.

Then \[
	\Sigma_\infty=\bigcap_{n=0}^\infty \Sigma_n
\] is also closed.

Then, $p\in \Sigma_\infty$ implies
that $f^{(n)}(p)=0\forall n$, which
implies $f(z)=0$ near $p$.  Thus $\Sigma_\infty$ is closed.

\begin{lem}
	$\Sigma_\infty\subset U$ is open.
\end{lem}
\begin{proof}
	Let $p\in \Sigma_\infty\subset U$.  To show $\Sigma_\infty$ is open at $p$,
	we need only to find some $\epsilon>0$ such that $D(p,\epsilon)\subset \Sigma_\infty$.
	
	Since $p\in \Sigma_\infty$, $f^{(n)}(p)=0\forall n$.  Thus the power
	series expansion of $f$ at $p$ is identically zero, and thus for some small $\epsilon$,
	$f(z)=0\forall z\in D(p,\epsilon)$, and thus $D(p,\epsilon)\subset \Sigma_\infty$.
\end{proof}

The final result follows from noting that $\Sigma_\infty$ can only be both
open and closed in $U$ iff it is empty or equal to $U$.

\begin{thm}
	Let $U\subset\complex$ be connected, and let $f:U\to\complex$ be holomorphic.
	Suppose $\{p_n\}\to p$ in $U$ such that $p_n\neq p\forall n$,
	and suppose $f(p_n)=0\forall n$.  Then $f=0$ on all of $U$.
\end{thm}
\begin{proof}
	We want to show that all the derivatives of $f$ at $p$ are zero.
	But $f$ is continuous, so $f(p)=0$.  We claim that this means
	$f^{(n)}(p)=0$ for all $n\geq 1$; otherwise, $p$
	is the only zero of $f$ in some $D(p,\epsilon)$ with $\epsilon>0$
	by the previous result; this contradicts $p_n\to p$ with $p_n\neq p$
	and $f(p_n)=0$ (because some $p_n$ must be in $D(p,\epsilon)^*$).
	
	This proves that \[
		\Sigma_\infty=\{\alpha\in U:f^{(n)}(\alpha)=0\forall n\}
	\] is nonempty, since it contains $p$.
	
	If $\Sigma_\infty$ is both closed and open in some connected domain
	$U$, and is also nonempty, then $\Sigma_\infty=U$.
\end{proof}

\begin{cor}
	Suppose $f,g:U\to\complex$ are both holomorphic, with $U$ connected.
	Suppose $\gamma\subset U$ is some (piecewise $C^1$?) arc in $U$, and \[
		f|_\gamma=g|_\gamma.
	\] Then $f=g$ on $U$.
\end{cor}
\begin{proof}
	By the previous result, $f-g$ has infinitely close zeros (along $\gamma$),
	so $f-g=0$ on all of $U$.
\end{proof}

Example: $\sin 2x=2\sin x\cos x$, for $x\in\reals$.  This
implies $\sin(2z)=2\sin z\cos z$ for all $z\in\complex$,
since we can take any arc $\gamma\subset \reals\subset \complex$,
and use the previous corollary.

Expanding on connectedness:

$U$ is connected iff for all $p,q\in U$, there is some arc $\gamma$
connecting $p$ and $q$.  Take some subset $V\subset U$
that is both closed and open in $U$, then assume $p\in V$, and
assume for contradiction $q\notin V$.  Since $\gamma$ is continuous,
the preimage $\gamma^{-1}(U)$ is both closed and open in $[0,1]$,
but the only such set is all of $[0,1]$; i.e. $q\in V$,
which is a contradiction.  Thus, either $V$ contains both
$p$ and $q$, or neither; thus $V$ is empty or $V=U$.

\begin{thm}
	Riemann extension theorem (i.e. removable singularity): Suppose
	$f:D(a,r)^*\to\complex$ is holomorphic.  Suppose $\lim_{z\to a}f(z)=\alpha$
	exists in $\complex$.  Then $f$ extends to a holomorphic function on $D(a,r)$,
	i.e. there exists $\alt{f}:D(a,r)\to\complex$ holomorphic such that \[
		\alt{f}|_{D(a,r)^*}=f.
	\]
\end{thm}
\begin{proof}
	Note that there is only one possible extension, i.e. \[
		\alt{f}(z)=\begin{dcases*}
			\alpha=\lim_{w\to a}f(w)&if $z=a$\\
			f(z)&otherwise
		\end{dcases*}
	\] since $\alt{f}$ must be $C^0$, i.e. continuous.
	
	We need to show that $\alt{f}$ is holomorphic.  One way
	is to show that \[
		\alt{f}'(a)=\lim_{h\to 0}\frac{\alt{f}(a+h)-\alt{f}(a)}{h}
	\] exists, but this may be difficult.
	
	Instead, use a power series expansion, or rather prove the result
	for a modified function with similar power series.  In particular,
	show $g(z)=(z-a)\alt{f}(z)$ is holomorphic on $D(a,r)^*$, which
	has no difficulty.  Then \[
		g'(a)=\lim_{h\to 0}\frac{g(a+h)-g(a)}{h}
		= \lim_{h\to 0}\frac{h\alt{f}(a+h)}{h}=\alt{f}(a)=\alpha,
	\] so $g$ is holomorphic.
	
	Then, either $g(z)=0$ for all $z\in D(a,r)$, in which
	case $\alt{f}=0$ is holomorphic; or $g$ is not identically zero,
	and $g(z)=(z-a)^k\alt{g}(z)$, with $\alt{g}(a)\neq 0$, $k\geq 1$
	(with $\alt{g}$ holomorphic).
	
	Then $\alt{f}(z)=\frac{g(z)}{z-a}=(z-a)^{k-1}\alt{g}(z)$ is holomorphic.
	
	Another alternative: Morera's theorem: If $f\in C^0(D(a,r))$ and $\oint_\gamma f=0$
	for all closed arcs $\gamma$, then $f$ is holomorphic on $D(a,r)$.
	The proof of this uses the existence of an antiderivative
	$F$ of $\alt{f}$ on a disc (using the taxicab-style path from $a$ to $z$
	for integration; this is well-defined iff integrals
	on closed paths are zero),
	after which direct calculation implies that the Cauchy-Riemann
	equations are satisfied, and $F$ is holomorphic.  Then,
	$\alt{f}=F'$ is holomorphic.
	
	Left to verify: $\alt{f}$ as defined has zero integral over every
	closed curve $\gamma$, and in particular every rectangular closed curve.
	Idea: over the integration path, cut the loop into two (if needed)
	pieces avoiding the point $a$, and use continuity at $a$ (we know
	closed curves not surrounding $a$ will have zero integral by holomorphicity
	of $f$).
\end{proof}
\label{lastpage}
\end{document}
