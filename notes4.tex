% Figures within a column...
\makeatletter
\newenvironment{tablehere}
{\def\@captype{table}}
{}
\newenvironment{figurehere}
{\def\@captype{figure}}
{}
\makeatother

\documentclass{artikel3}
\usepackage{amssymb,amsmath,amsthm,graphicx,enumerate,textcomp,color,datetime,mathtools}
\usepackage[left=1in,top=1in,right=1in,bottom=1in]{geometry}
\usepackage{fancyhdr}
\pagestyle{fancy}
\usepackage{pgfplots}
\yyyymmdddate
\date{\formatdate{2012}{04}{09}}
\pgfplotsset{compat=1.3}
\setlength{\headheight}{15pt}
\lhead{MATH 116: Lecture Notes 4, {\thepage} of \pageref{lastpage}}
\chead{}
\rhead{\today}
\lfoot{}
\cfoot{}
\rfoot{}
\newcommand{\pr}[1]{\left(#1\right)}
\newcommand{\abs}[1]{\left|#1\right|}
\renewcommand{\vec}[1]{\underline{#1}}
\newcommand{\alt}[1]{\tilde{#1}}
\newcommand{\dotp}[2]{#1\cdot #2}
\newcommand{\norm}[1]{\left\|#1\right\|}
\newcommand{\naturals}{\mathbb{N}}
\newcommand{\integers}{\mathbb{Z}}
\newcommand{\reals}{\mathbb{R}}
\newcommand{\complex}{\mathbb{R}}
\newcommand{\rationals}{\mathbb{Q}}
\newcommand{\eqlabel}[1]{\tag{\theequation}\label{#1}\refstepcounter{equation}}
\newcommand{\mseq}[2]{#1_1,\ldots,#1_{#2}}
\newcommand{\floor}[1]{\left\lfloor #1 \right\rfloor}
\newcommand{\ceil}[1]{\left\lceil #1 \right\rceil}
\newcommand{\mvd}{\twoheadrightarrow}
\newcommand{\unit}[1]{\;\rm{#1}}
\newcommand{\del}{\partial}
\newcommand{\indentblock}[1]{\addtolength{\leftskip}{5mm}
#1
}
\newcommand{\interior}[1]{\breve{#1}}
\newtheorem{lem}{Lemma}
\newcommand{\lemref}[1]{Lemma \ref{#1}}
\newcommand{\TODO}[1]{}
\DeclareMathOperator{\diam}{diam\;}
\begin{document}
\numberwithin{equation}{enumi}
\newcommand{\eqcite}[1]{\text{[#1]}}
\newcommand{\eqnref}[1]{Stmt. \ref{#1}}
\newcommand{\expval}[1]{\left\langle #1\right\rangle}
\newcommand{\closure}[1]{\overline{#1}}
\section{Complex Line Integrals}
Let $U\subset \complex$ be an open subset, $f:U\to \complex$ is a continuous
function, and $\gamma:[a,b]\to U$ is a (continuous) arc (i.e. \[
	\gamma:t\mapsto (x(t),y(t))\equiv x(t)+iy(t)\equiv z(t).
\] The line integral \[
	\int_{\gamma}f(z)dz \equiv \int_{a}^b \frac{d}{dt}f(z(t))\dot{z}(t)dt
\] (note the alternative notation $dz(t)\equiv \dot{z}(t)dt$).  Alternatively,
we can expand $f=u+iv$, to yield (in real line integral form) \[
	\int_\gamma (udx-vdy)+i\int_\gamma (vdx+udy).
\] Green's theorem for $\complex$ line integrals is as follows:

Theorem. Let $U$ be a simply connected region, and $f:U\to\complex$ is holomorphic.
Then \[
	\int_\gamma f(z)dz
\] is path-independent (i.e. it depends only on the endpoints $\gamma(a),\gamma(b)$,
not on the path in between).

Theorem. Let $U$ be a simply connected region, and let $f:U\to \complex$ be holomorphic.
Then there is a holomorphic function $g:U\to\complex$ such that \[
	\frac{\del}{\del z}g(z)=f(z)
\] (i.e. antiderivatives over simply connected regions exist).

Proof. Let $\alpha\in U$ be a fixed point.  For any $z\in U$, pick
a continuous arc $\gamma_z$ running from $\alpha$ to $z$, and along
$\gamma$, define $g$ by the definite/explicit path integral \[
	g(z)=\int_{\gamma_z} f(w)dw,
\] where $\gamma_z(a)=\alpha$ and $\gamma_z(b)=z$.

Claim. This integral is well-defined.  Justification: From previous theorem,
this integral is path-independent (since $U$ is simply connected, $f$ is holomorphic),
and thus attains only one value for all valid
$\gamma_z$ (which start at $\alpha$ and end at $z$).

Check: \[
	\frac{\del}{\del x}g(z)=f(z),\frac{\del}{\del y}g(z)=if(z).
\] For the former equation, consider the particular choice of $\gamma_z$
where, for $t\in [c,b]$, $\gamma_z(t)=x_0+(x-x_0)\frac{t-c}{b-c}+iy$, where $z=x+iy$
(i.e. $\gamma_z$ is the combination of a curve from $\alpha$ to $x_0+iy$
and a straight line segment from $x_0+iy$ to $x+iy$).
Denote by $\gamma_1$ the restriction of $\gamma_z$ to $[a,c]$ (i.e. the first
part), and
denote by $\gamma_2$ the function \[
	\gamma_2(t)=x_0+(x-x_0)\frac{t-c}{b-c}+iy,
\] i.e. $\gamma_2$ runs from $x_0+iy$ to $x+iy$.  Then \[
	g(z)=\int_{\gamma_1}f(w)dw+\int_{\gamma_2}f(w)dw
	= \int_{\gamma_1}f(w)dw+\int_{x_0}^{x}f(t+iy)dt,
\] so \[
	\frac{\del}{\del x}g(z)= f(x+iy)=f(z).
\]

For the second equation, consider some $z=x+iy$, and fix some
$y_0$.  Then, consider the particular curves $\gamma_1$ from
$\alpha$ to $x+iy_0$, and the curve $\gamma_2$ running from
$x+iy_0$ to $x+iy=z$.  Then by path independence, if $\gamma_z$
is any curve from $\alpha$ to $z$, \[
	\int_{\gamma}f(w)dw
	= \int_{\gamma_1}f(w)dw + \int_{\gamma_2}f(w)dw
	= \int_{\gamma_1}f(w)dw + \int_{y_0}^{y}f(x+it)idt,
\] so \[
	\frac{\del}{\del y}\int_{\gamma}f(w)dw = if(x+iy)=if(z).
\]

Then $g_x+ig_y=0$, so $g$ is holomorphic.  Furthermore, \[
	\frac{\del}{\del z}g(z)
	= \frac{1}{2}\pr{g_x(z)-ig_y(z)} = f(z).
\]

Weak extension of theorem:
Let $U$ be a disk, and $p\in U$.  Suppose $f:U\to \complex$ is continuous,
and $f$ is holomorphic on $U\setminus\{p\}$.  Then there is a holomorphic $g:U\to\complex$
such that \[
	\frac{\del}{\del z}g(z)=f(z)
\]

Proof: Fix $\alpha\in U$, and define \[
	g(z)=\int_{\gamma} f(w)dw,
\] where $\gamma$ runs from $\alpha$ to $z$.
Step 1: Show that $g(z)$ is also path-independent.

Step 2: verify $g$ is $C^1$, and verify that
$\del_x g(z)=f(z)$ and $\del_y g(z)=if(z)$ (note that
$\del_x\equiv \frac{\del}{\del x}$ and similarly
for $y$).

Step 1:  Since $U$ is an open disk, $\int_\gamma f(z)dz$ is independent of paths
iff for any simple closed arc $\Gamma\subset U$, $\int\Gamma f(z)dz=0$.
This makes sense since for any two paths $\gamma_1,\gamma_2$ with common endpoints,
$\gamma_1$ concatenated with the inverse of $\gamma_2$ forms a closed arc,
with path integral \[
	\int_{\gamma}f(z)dz = \int_{\gamma_1}f(z)dz-\int_{\gamma_2}f(z)dz,
\] so the integrals are the same iff $\int_{\gamma}f(z)dz$.

Caveat: the two curves $\gamma_1$ and $\gamma_2$ may intersect in complex manner, but
we can use continuity to perturb the two arcs $\gamma_1$ and $\gamma_2$ slightly,
so that they are non-intersecting and their combination is simple.  The difference between
this integral and the original is negligible (i.e. it is the integral over the tiny line
segments added to cover the perturbation), and so is arbitrarily small; thus the difference
between the integrals can satisfy this only if it is zero.

Tactic: This result is trivial for any curves not surrounding the excised point $P$.
For curves that do surround $P$, create two curves, $\gamma_-$ and $\gamma_+$, which each
cover one half of the area, excepting a very small $\epsilon$-sized region around $P$.  Thus,
if $\Gamma$ is the starting closed arc, \[
	\int_{\Gamma_2}f(z)dz = \int_{\gamma_1}f(z)dz + \int_{\gamma_2}f(z)dz
		+ \int_{\del B_\epsilon(p)}f(z)dz.
\] The first two terms are closed curves not surrounding $P$, and are zero by the previous result.
For the $\del B_\epsilon(p)$ term, \[
	\abs{\int_{\del B_\epsilon(p)}f(z)dz}
	= \abs{\lim_{\epsilon\to 0}\int_{\del B_\epsilon(p)}f(z)dz}
	\leq \operatorname{len}(\del B_\epsilon(p))\cdot \max_{\del B_\epsilon(p)}\abs{f(z)},
\] and the maximum is bounded since $f$ is continuous, so as $\epsilon\to 0$, the curve
length around $P$ goes to zero, and this term also vanishes.

A similar approach for curves running through $P$ (shift slightly to one side and bound the error).

In general, define $g(z)=\int_\gamma f(z)dz$, with $\gamma$ running from $\alpha$ to $z$.
Is this integral path-independent?  To proceed, must show that it is.

Application: Let $f:U\to\complex$ where $U$ is a disk and $f$ is holomorphic.
Define \[
	h(z)=\begin{dcases*}
		\frac{f(z)-z_0}{z-z_0}&where $z\neq z_0$.\\
		\frac{\del}{\del z}f(z)&where $z=z_0$.
	\end{dcases*}
\] Next time: define $f'(z_0)=\lim_{z\to z_0}\frac{f(z)-f(z_0)}{z-z_0}$ when the
limit exists (not always).

\label{lastpage}
\end{document}
