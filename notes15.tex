% Figures within a column...
\makeatletter
\newenvironment{tablehere}
{\def\@captype{table}}
{}
\newenvironment{figurehere}
{\def\@captype{figure}}
{}
\makeatother

\documentclass{artikel3}
\usepackage{amssymb,amsmath,amsthm,graphicx,enumerate,textcomp,color,datetime,mathtools}
\usepackage[left=1in,top=1in,right=1in,bottom=1in]{geometry}
\usepackage{fancyhdr}
\pagestyle{fancy}
\usepackage{pgfplots}
\yyyymmdddate
\date{\formatdate{2012}{05}{04}}
\pgfplotsset{compat=1.3}
\setlength{\headheight}{15pt}
\lhead{MATH 116: Lecture Notes 15, {\thepage} of \pageref{lastpage}}
\chead{}
\rhead{\today}
\lfoot{}
\cfoot{}
\rfoot{}
\newcommand{\pr}[1]{\left(#1\right)}
\newcommand{\abs}[1]{\left|#1\right|}
\renewcommand{\vec}[1]{\underline{#1}}
\newcommand{\alt}[1]{\tilde{#1}}
\newcommand{\dotp}[2]{#1\cdot #2}
\newcommand{\norm}[1]{\left\|#1\right\|}
\newcommand{\naturals}{\mathbb{N}}
\newcommand{\integers}{\mathbb{Z}}
\newcommand{\reals}{\mathbb{R}}
\newcommand{\complex}{\mathbb{R}}
\newcommand{\rationals}{\mathbb{Q}}
\newcommand{\eqlabel}[1]{\tag{\theequation}\label{#1}\refstepcounter{equation}}
\newcommand{\mseq}[2]{#1_1,\ldots,#1_{#2}}
\newcommand{\floor}[1]{\left\lfloor #1 \right\rfloor}
\newcommand{\ceil}[1]{\left\lceil #1 \right\rceil}
\newcommand{\mvd}{\twoheadrightarrow}
\newcommand{\unit}[1]{\;\rm{#1}}
\newcommand{\del}{\partial}
\newcommand{\indentblock}[1]{\addtolength{\leftskip}{5mm}
#1
}
\newcommand{\interior}[1]{\breve{#1}}
\newtheorem{lem}{Lemma}
\newtheorem{thm}{Theorem}
\newtheorem{cor}{Corollary}
\newtheorem{prop}{Proposition}
\newcommand{\lemref}[1]{Lemma \ref{#1}}
\newcommand{\TODO}[1]{}
\DeclareMathOperator{\diam}{diam\;}
\numberwithin{equation}{enumi}
\newcommand{\eqcite}[1]{\text{[#1]}}
\newcommand{\eqnref}[1]{Stmt. \ref{#1}}
\newcommand{\expval}[1]{\left\langle #1\right\rangle}
\newcommand{\closure}[1]{\overline{#1}}
\begin{document}
(Late to class, missed :( )

One solution: restrict to a smaller open set $U$ such that
a continuous inverse (e.g. $\log z$, $e^{\alpha\log z}$)
can be defined.

(Erased! :[ )

If $f$ is continuously defined up to the boundary of $U$,
we can speak of integration over the boundary (where the
integral can then be calculated by the sum of the residues
inside $U$).

Apply this to $f(z)=\frac{z^{1/4}}{1+z^2}$ where $U=\{\abs{z}<R\}\setminus [0,R]$.

Note that $f$ is holomorphic on $U$, $U$ is bounded, and $f$ is $C^0$
up to $\del U$ with the understanding that $f$ has different
values on the upper and lower sides of $[0,R]$.  Note that
the value of $f$ on the outer circle $\Gamma_R$ of radius $R$
is unambiguous.

Calculating, if $\gamma_+$ is the line $[0,R]$ on the ``upper side''
of the boundary, \[
	f|_{\gamma_+}=\frac{r^{1/4}}{1+r^2},
\] but for $\gamma_-$ on the lower side, \[
	f|_{\gamma_-}=\frac{(re^{i2\pi})^{1/4}}{1+(re^{i2\pi})^2}=\frac{r^{1/4}e^{i\pi/2}}{1+r^2}.
\]

To make this rigorous, consider this taken as a limit of curves
near, but not on $[0,R]$, and take limits.

For example: consider $\int_0^\infty \frac{dx}{x^2+6x+8}$.
Note first that $x^2+6x+8=(x+4)(x+2)$.  Note that the usual
method of integrating over a large circle does not work,
because we are integrating from $0$ to $\infty$, not $-\infty$
to $\infty$.

Trick: create a multi-valued function for the purposes of this integral.
Consider \[
	\frac{1}{2\pi i}\oint_{\del U}\frac{\log z}{z^2+6z+8}dz.
\] As before, \[
	\log z|_{\gamma_+}=\log r
\] and \[
	\log z|_{\gamma_-}=\log r+2\pi i.
\] The $\log r$ cancels in integration, the outer circle
integrates to zero, and the difference between
the $\gamma_+$ and $\gamma_-$ integrals (the $2\pi i$ term)
creates the desired result.

Exercise 62.  Prove \[
	\sum_{k=-\infty}^{\infty}\frac{1}{(k+\alpha)^2}=\frac{\pi^2}{\sin^2\pi\alpha}
\] where $\alpha\notin \integers$.

Solution.  Via experience, look at \[
	\frac{1}{(z+\alpha)^2}\frac{\cos \pi z}{\sin\pi z}.
\] This is based on needing every integer $z$ to be a pole, and adding
a factor $\frac{1}{(z+\alpha)^2}$ to make the value of the residue
correct.  This introduces another pole of order $2$ at $z=-\alpha$.

Use an integral over a large box, sum the residues and solve
(ideally, $\frac{\pi^2}{\sin^2\pi\alpha}$ will arise from
the residue for $z=-\alpha$, while the sum will arise from
the residues at each $z\in\integers$.

In particular, take $R$ to be a non-integer (i.e. integer
plus $\frac{1}{2}$ so that $\frac{1}{(z+\alpha)^2}~\frac{1}{R^2}$
and $\abs{\frac{\cos \pi z}{\sin \pi z}}$ is bounded).

Meromorphic function.  Take $U\subset \complex$ open.

Definition. A subset $S\subset U$ is discrete if for any $p\in S$
there is some $r>0$ such that $D(p,r)^*\subset U\setminus S$
(equivalently, $p$ is not a limit point of $S$).

Definition. $f$ is meromorphic on $U$ (an open
subset of $\complex$) iff there is a closed, discrete
subset $S\subset U$ such that
\begin{enumerate}
	\item
		$f:U\setminus S\to \complex$ is holomorphic.
	\item
		For every $p\in S$, $p$ is either a removable
		singularity or a pole of $f$ (i.e. not an essential
		singularity).
\end{enumerate}

Example: $\frac{p(z)}{q(z)}$ where $p,q$ polynomials
with no common factors is meromorphic on $\complex$.

Definition.  The extended complex plane $\bar{C}=\complex\cup\{\infty\}$
is defined such that a (punctured) neighborhood of $\infty$ is a set
$\{z\in\complex:\abs{z}>R\}$ for some $R\in (0,\infty)$.

Intuitively, think of the complex plane as the stereographic projection
of a sphere less a point; $\infty$ covers this extra point to form a full
sphere.

Using this definition, consider $f:U\to\complex$ holomorphic.
We say $\infty$ is an isolated singularity of $f$ if there is
an $R$ such that $\{\abs{z}>R\}\subset U$.

Definition. We say $f:U\to\complex$ has a removable singularity
(resp. pole; resp. essential singularity) at $\infty$ if
\begin{enumerate}
	\item
		$\infty$ is an isolated singularity of $f$.
	\item
		$f(1/z)$ has a removable singularity (resp. pole; resp. essential singularity)
		at $0$.
\end{enumerate}

We say that meromorphic $f$ on $\complex$ is meromorphic at $\infty$
(or meromorphic on $\bar{\complex}$) if $\infty$ is an isolated
singularity and is not an essential singularity.

Example: A rational function $f(z)=\frac{p(z)}{q(z)}$ with
$p,q$ polynomials with no common factors is a meromorphic
function on $\bar{\complex}$.

\begin{thm}
	A meromorphic function on $\bar{\complex}$ is a rational function.
\end{thm}
This theorem is the last problem on the assignment.

Strategy:
First argue that there are only finitely many poles (use the isolated
point property on $\bar{C}$).

Now take $(z-\alpha_1)\cdots (z-\alpha_n)f(z)$, which no longer
has poles at $\complex$.  At this point, either we can use Liouville
or there must be a pole
at $\infty$, which means there are only finitely many zeros inside
$\complex$.  Divide out the zeros and repeat.  If there is still a pole at
$\infty$, take $1/f(z)$ to achieve a contradiction.
\label{lastpage}
\end{document}
