% Figures within a column...
\makeatletter
\newenvironment{tablehere}
{\def\@captype{table}}
{}
\newenvironment{figurehere}
{\def\@captype{figure}}
{}
\makeatother

\documentclass{artikel3}
\usepackage{amssymb,amsmath,amsthm,graphicx,enumerate,textcomp,color,datetime,mathtools}
\usepackage[left=1in,top=1in,right=1in,bottom=1in]{geometry}
\usepackage{fancyhdr}
\pagestyle{fancy}
\usepackage{pgfplots}
\yyyymmdddate
\date{\formatdate{2012}{05}{02}}
\pgfplotsset{compat=1.3}
\setlength{\headheight}{15pt}
\lhead{MATH 116: Lecture Notes 14, {\thepage} of \pageref{lastpage}}
\chead{}
\rhead{\today}
\lfoot{}
\cfoot{}
\rfoot{}
\newcommand{\pr}[1]{\left(#1\right)}
\newcommand{\abs}[1]{\left|#1\right|}
\renewcommand{\vec}[1]{\underline{#1}}
\newcommand{\alt}[1]{\tilde{#1}}
\newcommand{\dotp}[2]{#1\cdot #2}
\newcommand{\norm}[1]{\left\|#1\right\|}
\newcommand{\naturals}{\mathbb{N}}
\newcommand{\integers}{\mathbb{Z}}
\newcommand{\reals}{\mathbb{R}}
\newcommand{\complex}{\mathbb{R}}
\newcommand{\rationals}{\mathbb{Q}}
\newcommand{\eqlabel}[1]{\tag{\theequation}\label{#1}\refstepcounter{equation}}
\newcommand{\mseq}[2]{#1_1,\ldots,#1_{#2}}
\newcommand{\floor}[1]{\left\lfloor #1 \right\rfloor}
\newcommand{\ceil}[1]{\left\lceil #1 \right\rceil}
\newcommand{\mvd}{\twoheadrightarrow}
\newcommand{\unit}[1]{\;\rm{#1}}
\newcommand{\del}{\partial}
\newcommand{\indentblock}[1]{\addtolength{\leftskip}{5mm}
#1
}
\newcommand{\interior}[1]{\breve{#1}}
\newtheorem{lem}{Lemma}
\newtheorem{thm}{Theorem}
\newtheorem{cor}{Corollary}
\newtheorem{prop}{Proposition}
\newcommand{\lemref}[1]{Lemma \ref{#1}}
\newcommand{\TODO}[1]{}
\DeclareMathOperator{\diam}{diam\;}
\numberwithin{equation}{enumi}
\newcommand{\eqcite}[1]{\text{[#1]}}
\newcommand{\eqnref}[1]{Stmt. \ref{#1}}
\newcommand{\expval}[1]{\left\langle #1\right\rangle}
\newcommand{\closure}[1]{\overline{#1}}
\begin{document}
Residue theorem allows us to calculate definite integrals for closed loops.
Remark: \[
	\int_{-\infty}^{\infty}\equiv \lim_{R\to\infty}\int_{[-R,R]}.
\] Combining this with arc $\Gamma_R^+$ (the upper semicircle
with radius $R$), \[
	\int_{[-R,R]}+\int_{\Gamma_R^+}=2\pi i\sum\mathrm{Residues}\in\complex.
\] Verify: $\lim_{R\to\infty}\int_{\Gamma_R^+}=0$, which
then implies $\int_{-\infty}^\infty = 2\pi i\sum\mathrm{Residues}$.

Using the example $\frac{1}{1+z^4}$, note that \[
	\mathrm{Res}(e^{i\pi/4})=\lim_{z\to e^{i\pi/4}}\frac{z-e^{i\pi/4}}{1+z^4}=\frac{1}{4z^3}=\frac{1}{4}e^{-i3\pi/4}.
\] Similarly calculate residue for $e^{i3\pi/4}$, and plug in.

For quick check, \[
	\int_{\Gamma_R^+}\frac{1}{1+z^4}dz\to 0
\] as $R\to\infty$ because the degree of $(1+z^4)^{-1}$ is $-4$.

Remark: $\int_{-\infty}^\infty \frac{q(x)}{p(x)}dx$ when $p(x)$ has no
real roots and the degree of $q(x)$ is at most degree of $p(x)$ minus $2$
can be calculated this way.

Example: $\int_0^{2\pi} \frac{d\theta}{a+\sin\theta}$ for $a>1$.
Strategy: note that $\theta$ ranging from $0$ to $2\pi$ is like
a parametrization of the unit circle.  Take $z(\theta)=e^{i\theta}$.
Then $dz(\theta)=ie^{i\theta}d\theta$, so $d\theta=\frac{1}{ie^{i\theta}}dz=\frac{dz}{iz}$.
Thus \[
	\int_{\abs{z}=1}\frac{dz/(iz)}{a+\frac{z-z^{-1}}{2i}}
	=\int_{\abs{z}=1}\frac{2dz}{z^2+2iaz-1}.
\] Now, apply the residue theorem, and consider the poles;
by calculation \[
	z=-ia\pm\sqrt{1-a^2}.
\] Considering these roots, note that the product of the two roots
is $1$, so one of the two roots is inside, and one is outside
the loop.  The roots can be rewritten as \[
	z=i(-a\pm\sqrt{a^2-1}),
\] where $i(-a+\sqrt{a^2-1})$ is inside the loop.
Thus \[
	\int_0^{2\pi} \frac{d\theta}{a+\sin\theta}=2\pi i\mathrm{Res}(-ia+i\sqrt{a^2-1}).
\] In general, this substitution $z=e^{i\theta}$ can be used to calculate closed
loop (circular) integrals of rational functions on $\sin\theta$, $\cos\theta$.

Exercise 47.  Calculate $\int_{-\infty}^{\infty}\frac{\cos x}{1+x^4}dx$.
First, try $\int_{[-R,R]}+\int_{\Gamma_R^+}$, as before.

Looking at \[
	\int_{\Gamma_R^+}\frac{1}{2}\frac{e^{iz}+e^{-iz}}{1+z^4}dz,
\] note that the latter term $e^{-iz}$ explodes as $z\to i\infty$.
Thus, this approach will not work.  Instead, use \[
	\int_{-\infty}^\infty \frac{\cos x}{1+x^4}dx
	=\int_{-\infty}^\infty \frac{\Re(e^{ix})}{1+x^4}dx
	=\Re\pr{\int_{-\infty}^\infty \frac{e^{ix}}{1+x^4}dx}.
\] Now, instead integrate over the boundary of the rectangle
with vertices $R,R+iR,-R+iR,-R$.  Then, \[
	\int_{R\to R+iR}\frac{e^{iz}}{1+z^4}dz=
	\int_{0}^{R}\frac{e^{iR-t}}{1+(R+it)^4}dz,
\] where the top is bounded and the denominator is of order $R^4$,
so the integral goes to zero.  The other branches
can be checked similarly.

Another note: in general, we can say \[
	\int_{0}^{\infty}\frac{dx}{1+x^2}=\frac{1}{2}\int_{-\infty}^{\infty}\frac{dx}{1+x^2},
\] but the same trick does not work for \[
	\int_{0}^{\infty}\frac{x^{1/4}}{1+x^2}dx.
\] Now, define $z^{1/4}$ holomorphic over $D(0,R)\setminus[0,R)$
(i.e. take a sheaf).  Take our loop integral where
\begin{align*}
	\gamma_1(t)&=t,t\in [0,R],\\
	\gamma_2(t)&=Re^{i\theta},\theta\in[0,2\pi],\\
	\gamma_3(t)&=se^{2\pi i},s\in [R,0].
\end{align*}
We write $\gamma_3$ as such to denote a slight gap between $\gamma_1$ and $\gamma_3$,
over the break in the definition of $z^{1/4}$.
Take \[
	\int_{\gamma_1+\gamma_2+\gamma_3}\frac{z^{1/4}}{1+z^2}dz
	= \int_{0}^{R}\frac{t^{1/4}}{1+t^2}dt+\int_{0}^{2\pi}\frac{R^{1/4}e^{i\theta/4}}{1+R^2e^{2i\theta}}iRe^{i\theta}d\theta
	+\int_{R}^{0}\frac{(se^{2\pi i})^{1/4}}{1+(se^{2\pi i})^2}ds.
\] Note that the $\gamma_2$ integral goes to zero, because $R^2$ dominates.
Thus \[
	\mathrm{RHS}=\int_{0}^R \frac{t^{1/4}}{1+t^2}-\int_{0}^{R}\frac{e^{i\pi/2}s^{1/4}}{1+s^2}ds,
\] Thus \[
	2\pi i\sum\mathrm{Res}=(1-e^{2\pi i})I,
\] where $I$ is the desired integral.
\label{lastpage}
\end{document}
