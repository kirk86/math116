% Figures within a column...
\makeatletter
\newenvironment{tablehere}
{\def\@captype{table}}
{}
\newenvironment{figurehere}
{\def\@captype{figure}}
{}
\makeatother

\documentclass{artikel3}
\usepackage{amssymb,amsmath,amsthm,graphicx,enumerate,textcomp,color,datetime,mathtools}
\usepackage[left=1in,top=1in,right=1in,bottom=1in]{geometry}
\usepackage{fancyhdr}
\pagestyle{fancy}
\usepackage{pgfplots}
\yyyymmdddate
\date{\formatdate{2012}{04}{18}}
\pgfplotsset{compat=1.3}
\setlength{\headheight}{15pt}
\lhead{MATH 116: Lecture Notes 8, {\thepage} of \pageref{lastpage}}
\chead{}
\rhead{\today}
\lfoot{}
\cfoot{}
\rfoot{}
\newcommand{\pr}[1]{\left(#1\right)}
\newcommand{\abs}[1]{\left|#1\right|}
\renewcommand{\vec}[1]{\underline{#1}}
\newcommand{\alt}[1]{\tilde{#1}}
\newcommand{\dotp}[2]{#1\cdot #2}
\newcommand{\norm}[1]{\left\|#1\right\|}
\newcommand{\naturals}{\mathbb{N}}
\newcommand{\integers}{\mathbb{Z}}
\newcommand{\reals}{\mathbb{R}}
\newcommand{\complex}{\mathbb{R}}
\newcommand{\rationals}{\mathbb{Q}}
\newcommand{\eqlabel}[1]{\tag{\theequation}\label{#1}\refstepcounter{equation}}
\newcommand{\mseq}[2]{#1_1,\ldots,#1_{#2}}
\newcommand{\floor}[1]{\left\lfloor #1 \right\rfloor}
\newcommand{\ceil}[1]{\left\lceil #1 \right\rceil}
\newcommand{\mvd}{\twoheadrightarrow}
\newcommand{\unit}[1]{\;\rm{#1}}
\newcommand{\del}{\partial}
\newcommand{\indentblock}[1]{\addtolength{\leftskip}{5mm}
#1
}
\newcommand{\interior}[1]{\breve{#1}}
\newtheorem{lem}{Lemma}
\newtheorem{thm}{Theorem}
\newcommand{\lemref}[1]{Lemma \ref{#1}}
\newcommand{\TODO}[1]{}
\DeclareMathOperator{\diam}{diam\;}
\numberwithin{equation}{enumi}
\newcommand{\eqcite}[1]{\text{[#1]}}
\newcommand{\eqnref}[1]{Stmt. \ref{#1}}
\newcommand{\expval}[1]{\left\langle #1\right\rangle}
\newcommand{\closure}[1]{\overline{#1}}
\begin{document}
\section{Power series expansion}
Consider $f:U\to \complex$ holomorphic, and $D(a,r)\subset U$.

\begin{thm}
	$f(z)|_{D(a,r)}=f(a)+f'(a)(z-a)+\cdots +\frac{f^{(n)}(a)}{n!}(z-a)^n+\cdots$
	where the power series converges over $D(a,r)$, and $=$ holds on $D(a,r)$ too.
\end{thm}

\begin{proof}
	Take $z\in D(a,r)$, and take $r'$
	such that $z\in D(a,r')\subset \overline{D(a,r')}\subset D(a,r)$.  Then \[
		f(z)=\frac{1}{2\pi i}\oint_{\abs{w-a}=r'}\frac{f(w)}{w-z}dw.
	\] Note that \[
		\frac{1}{w-z}=\frac{1}{(w-a)-(z-a)}=\frac{1}{w-a}\frac{1}{1-\frac{z-a}{w-a}}.
	\] Observation: in the integral, $\abs{w-a}=r'$, $\abs{z-a}\leq r'-\epsilon$
	for some $\epsilon>0$.  Then \[
		\abs{\frac{z-a}{w-a}}\leq \frac{r'-\epsilon}{r'}=1-\frac{\epsilon}{r'}<1.
	\] Thus \[
		\frac{1}{1-\frac{z-a}{w-a}}=\sum_{n=0}^{\infty}\pr{\frac{z-a}{w-a}}^n.
	\] Thus \[
		\frac{1}{w-z}=\sum_{n=0}^{\infty}\frac{(z-a)^n}{(w-a)^{n+1}}
	\] converges uniformly for $\abs{w-a}=r'$, $\abs{z-a}\leq 1-\frac{\epsilon}{r'}$.

	By the Cauchy integral formula, \[
		f(z)=\frac{1}{2\pi i}\oint_{\abs{w-a}=r}\frac{f(w)}{w-z}dw
		=\frac{1}{2\pi i}\oint_{\abs{w-a}=r'}\pr{f(w)\sum_{n=0}^{\infty}\frac{(z-a)^n}{(w-a)^{n+1}}}dw.
	\] Due to the uniform convergence, we can extract the sum from the integral, so \[
		f(z)=\frac{1}{2\pi i}\sum_{n=0}^{\infty}\oint_{\abs{w-a}=r'}\pr{f(w)\frac{(z-a)^n}{(w-a)^{n+1}}}dw
		= \sum_{n=0}^{\infty}\pr{\frac{1}{2\pi i}\oint_{\abs{w-a}=r'}\frac{f(w)}{(w-a)^{n+1}}dw}(z-a)^n,
	\] where the interior integral is equal to $\frac{f^{(n)}(a)}{n!}$.  Thus \[
		f(z)=\sum_{n=0}^{\infty}\frac{f^{(n)}(a)}{n!}(z-a)^n.
	\]
\end{proof}

The argument implies that this series converges and is equal to $f(z)$ for
$\abs{z-a}\leq r'-\epsilon$, with $r'<r$ and $\epsilon>0$.  But for any $z\in D(a,r)$, we may choose
$r'$ and $\epsilon$ satisfactory.  Thus the series converges over all of $D(a,r)$.
General argument: pick a circle slightly smaller than $r$, use uniform
convergence and the Cauchy integral over the circle to translate the sum
outside the integral (also, take the Cauchy integral and expand it as a power
series with ratio $\frac{z-a}{w-a}$).

Note that the power series converges over all of $D(a,r)$, but converges uniformly
over any slightly smaller disc $D(a,r')$, and particularly over its closure as well.

Note also that uniform convergence allows for operations like differentiating under the sum?
(not sure here, didn't hear that properly)

Example: $e^z=e^x(\cos y+i\sin y)$.  $(e^z)'=e^z$, so $(e^z)^{(n)}|_{z=0}=1$.
By the above, \[
	e^z=1+z+\frac{z^2}{2}+\cdots+\frac{z^n}{n!}+\cdots.
\] Remark: using the root test, \[
	r=\frac{1}{\limsup_{\sqrt[n]{1/n!}}}=\infty,
\] since $n!\approx \pr{\frac{n}{e}}^n\sqrt{2\pi n}$.
Note that \[
	\sin z=\frac{1}{2i}(e^{iz}-e^{-iz}), \cos z=\frac{1}{2}(e^{iz}+e^{-iz})
\] are both holomorphic.  Additionally, \[
	\sinh z=\frac{1}{2}(e^z-e^{-z}), \cosh z=\frac{1}{2}(e^z+e^{-z}).
\] Note that almost all operations with the power series (sum, product, etc.)
can be done with the power series, due to its strong convergence (i.e. geometric).

\section{Cauchy Estimate}
\begin{thm}
	Take $f:U\to\complex$ holomorphic, and $\overline{D(a,r)}\subset U$, and \[
		M=\sup_{w\in\del D(a,r)}\abs{f(w)}.
	\] Then \[
		\abs{f^{(n)}(a)}\leq\frac{n!}{r^n}M.
	\]
\end{thm}
\begin{proof}
	This follows from the Cauchy integral.  The formula yields \[
		f^{(n)}(a)=\frac{n!}{2\pi i}\oint_{\abs{w-a}=r}\frac{f(w)}{(w-a)^{n+1}}dw.
	\] Thus \[
		\abs{f^{(n)}(a)}\frac{n!}{2\pi}\abs{\oint_{\abs{w-a}=r}\frac{f(w)}{(w-a)^{n+1}}dw}
		\leq \frac{n!}{2\pi}\oint_{\abs{w-a}=r}\abs{\frac{f(w)}{(w-a)^{n+1}}}\abs{dw}.
	\] Recall that $dw$ is complex, so it also requires an absolute value.  Thus \[
		\frac{n!}{2\pi}\oint_{\abs{w-a}=r}\abs{\frac{f(w)}{(w-a)^{n+1}}}\abs{dw}
		= \frac{n!}{2\pi}\int_{0}^{2\pi}\abs{\frac{f(a+re^{i\theta})}{(re^{i\theta})^{n+1}}}\abs{ire^{i\theta}}d\theta.
	\] Then, $\abs{f(a+re^{i\theta})}\leq M$, so \[
		\frac{n!}{2\pi}\int_{0}^{2\pi}\abs{\frac{f(a+re^{i\theta})}{(re^{i\theta})^{n+1}}}\abs{ire^{i\theta}}d\theta
		\leq \frac{n!}{2\pi}\int_{0}^{2\pi}\frac{M}{r^{n+1}}rd\theta=\frac{n!}{r^n}M.
	\]
\end{proof}

\begin{thm}
	Under same conditions, take $z\in D(a,r)$.  Then \[
		f^{(n)}(z)\leq \frac{rn!}{(r-\abs{z-a})^n}M.
	\]
\end{thm}
\begin{proof}
	Then the proof follows the same structure as the previous,
	except that the denominator becomes $\abs{re^{i\theta}-(z-a)}\geq r-\abs{z-a}$ instead
	of $\abs{re^{i\theta}}=r$.
	Thus \[
		\abs{f^{(n)}(z)}\leq \frac{rn!}{(r-\abs{z-a})^{n+1}}M.
	\]
\end{proof}

\section{Liouville Theorem}
\begin{thm}
	A bounded holomorphic function $f:\complex\to\complex$ is a constant
	function.
\end{thm}
\begin{proof}
	$f$ bounded iff $\abs{f(z)}\leq M$ for some $M$.  Then, take any large disc
	of radius $R$ centered at some point $a\in\complex$.  Applying the previous estimate, \[
		\abs{f'(a)}\leq \frac{M}{R}.
	\] As $R$ can be arbitrarily large, $\abs{f'(a)}=0$.  Thus, $f'=0$ identically
	for all $a\in\complex$, and $f$ is constant.
\end{proof}

\section{Gauss' Fundamental Theorem of Algebra}
\begin{thm}
	Any positive degree polynomial
	with complex coefficients (defined on $\complex$) has at least one root in $\complex$.
\end{thm}

\begin{proof}[Proof (one of many)]
	Suppose $p(z)=z^n+a_{n-1}z^{n-1}+\cdots +a_0$
	(WLOG assume leading coeff. $1$), with $n\geq 1$, and $p$ has no roots in $\complex$.
	Take \[
		f(z)=\frac{1}{p(z)}.
	\] Then, show that $f:\complex\to\complex$ is holomorphic.  Following that,
	show that $f(z)$ is bounded on $\complex$ (i.e. when $z$ is of very large
	magnitude, $z^n$ term dominates, so $p(z)$ also very large (and for smaller $z$,
	bound by compactness argument over sufficiently large closed ball).

	In particular, \[
		\frac{1}{p(z)}=\frac{1}{z^n}\frac{1}{1+a_{n-1}z^{-1}+\cdots+a_0z^{-n}}.
	\] Pick $R$ such that for $\abs{z}\geq R$, \[
		\abs{a_{n-k}z^{-k}}\leq \frac{1}{n+2}.
	\] Then $\abs{z}\geq R\geq 10(n+1)^2$, \[
		\abs{\frac{1}{p(z)}}\leq \abs{\frac{1}{z^n}}\frac{1}{1-\frac{n}{n+2}}\leq 1.
	\] Liouville Thm. implies $p$ constant, but this contradicts
	the positive degree of $p$.
\end{proof}
\label{lastpage}
\end{document}
